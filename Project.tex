\documentclass[margin,line]{res}
\usepackage{multirow}
\usepackage{graphicx}
\usepackage{float}
\usepackage{amsmath}
\usepackage{cases}
\usepackage{amsfonts}
\usepackage{stfloats}
\usepackage{cite}
\usepackage{color}
\usepackage{ctex}


\oddsidemargin -.5in
\evensidemargin -.5in
\textwidth=6.0in
\itemsep=0in
\parsep=0in
% if using pdflatex:
\setlength{\pdfpagewidth}{\paperwidth}
\setlength{\pdfpageheight}{\paperheight} 

\newenvironment{list1}{
  \begin{list}{\ding{113}}{%
      \setlength{\itemsep}{0in}
      \setlength{\parsep}{0in} \setlength{\parskip}{0in}
      \setlength{\topsep}{0in} \setlength{\partopsep}{0in} 
      \setlength{\leftmargin}{0.17in}}}{\end{list}}
\newenvironment{list2}{
  \begin{list}{$\bullet$}{%
      \setlength{\itemsep}{0in}
      \setlength{\parsep}{0in} \setlength{\parskip}{0in}
      \setlength{\topsep}{0in} \setlength{\partopsep}{0in} 
      \setlength{\leftmargin}{0.2in}}}{\end{list}}


\begin{document}

\name{Course Project for Signal and System  \vspace*{.1in}$\ \ \ \ \ \ \ \ \ \ \ \ \ \ \ \ \ \ \ \ \ \ \ \ \ \ \ \ \ \ \ \ \ \ \ \ \ \ \ \ \ \ \ \ \ \ \ \ \ \ $ \includegraphics[height=6em]{244}}


\begin{resume}
%\section{\sc Contact Information}
\vspace{.05in}
\begin{tabular}{@{}p{3.7in}p{4in}}           
	School of Communication and Information System,  & {\it Name:}  朱彦晟 \\         
	Xi'an University of Post and Telecommunications & {\it Identifier:}  05202009
\end{tabular}

\section{\sc I. Where the story begins}
\newpage

\section{\sc II. When the infinite sum has to be applied}
\newpage

\section{\sc III. Open the gate to frequency domain}
\newpage

\section{\sc IV. From periodicity to aperiodicity}
\newpage




\end{resume}
\end{document}




