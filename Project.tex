\let\nofiles\relax
\documentclass[margin,line]{res}
\usepackage{multirow}
\usepackage{graphicx}
\usepackage{float}
\usepackage{amsmath}
\usepackage{cases}
\usepackage{amsfonts}
\usepackage{stfloats}
\usepackage{cite}
\usepackage{color}
\usepackage{mathrsfs}
\usepackage{animate}

\oddsidemargin -.5in
\evensidemargin -.5in
\textwidth=6.0in
\itemsep=0in
\parsep=0in
% if using pdflatex:
\setlength{\pdfpagewidth}{\paperwidth}
\setlength{\pdfpageheight}{\paperheight} 

\newenvironment{list1}{
  \begin{list}{\ding{113}}{%
      \setlength{\itemsep}{0in}
      \setlength{\parsep}{0in} \setlength{\parskip}{0in}
      \setlength{\topsep}{0in} \setlength{\partopsep}{0in} 
      \setlength{\leftmargin}{0.17in}}}{\end{list}}
\newenvironment{list2}{
  \begin{list}{$\bullet$}{%
      \setlength{\itemsep}{0in}
      \setlength{\parsep}{0in} \setlength{\parskip}{0in}
      \setlength{\topsep}{0in} \setlength{\partopsep}{0in} 
      \setlength{\leftmargin}{0.2in}}}{\end{list}}

\newcommand*{\dif}{\mathop{}\!\mathrm{d}}

\begin{document}

\name{Course Project for Signal and System  \vspace*{.1in}$\ \ \ \ \ \ \ \ \ \ \ \ \ \ \ \ \ \ \ \ \ \ \ \ \ \ \ \ \ \ \ \ \ \ \ \ \ \ \ \ \ \ \ \ \ \ \ \ \ \ $\includegraphics[height=6em]{244}}


\begin{resume}
%\section{\sc Contact Information}
\vspace{.05in}
\begin{tabular}{@{}p{3.7in}p{4in}}           
	School of Communication and Information System,  & {\it Name:}  YanSheng Zhu \\         
	Xi'an University of Post and Telecommunications & {\it Identifier:}  05202009
\end{tabular}

\section{\sc I. Where the story begins}
When we play the piano,we can press a key,such as Do. Its waveform is like the second curve in the Figure 1. Also we can get La which like the first curve in Figure 1. Then we press these two keys at the same time. Then we will get the third curve in the Figure 1.\par
But when we get a signal which like the third curve, we want to know which two buttons generate it. We want to know how to build this signal.\par
Generally we want to decompose a signal, which like the 4th curve in the Figure 1, into the combination of some simple signal.\par
\begin{figure}[H]
  \begin{minipage}{0.42\linewidth}
    \centerline{\includegraphics[width=0.9\textwidth]{figure/fig_1.png}}
    \centerline{\textbf{Figure 1}}
  \end{minipage}
  \begin{minipage}{0.55\linewidth}
    \centerline{\includegraphics[width=\textwidth]{figure/fig_2.png}}
    \centerline{\textbf{Figure 2}}
  \end{minipage}
\end{figure}
In fact, many physical problems have the same properties, such as heat transfer problem. If we have a rod, which temperature function is $T=cos(x)$ .
The change of its temperature with time t presents a simple exponential decay form, as I showed in the first sub-graph of Figure 2.\par
But for a more complex temperature function $T=f(x)$ like the second sub-graph of Figure 2, if we can decompose it into the composition of a series sine and cosine function, we can simply migrate it to $T=F(x,t)$.\par
Let's start with simple one:a sinusoidal signal of 4 beats per second. We can construct a rotating vector whose length is the same as this sinusoidal signal value of the current time. The graph of this vector will like the second sub-graph in Figure 3. We can find a magic phenomenon.When the frequency of this rotating vector is equal to this sinusoidal signal's frequency, a large number of points in this image are biased to the same side.\par
If we imagine this graph is having some kind of mass to it, like metal wire. We can find, as we change the frequency, the center of mass wobbles around a bit at origin. But when the frequency is equal to our signal frequency, the mass center is unusually far to the right. If we draw the x position of the mass center changed by the frequency into a $f-x$ graph. It will be like the third sub-figure in Figure 3. There is a high peak at 4 Hz position.\par
\begin{figure}[H]
	\begin{minipage}{0.48\linewidth}
		\centerline{\animategraphics[autoplay,loop,width=0.8\textwidth]{30}{figure/fig_3/fig-}{1}{601}}
		\centerline{\textbf{Figure 3}}
	\end{minipage}
	\begin{minipage}{0.48\linewidth}
		\centerline{\animategraphics[autoplay,loop,width=0.8\textwidth]{30}{figure/fig_4/fig-}{1}{601}}
		\centerline{\textbf{Figure 4}}
	\end{minipage}
\end{figure}
And we can do the same operation in a combination of two signal: $cos(2\pi 3t)+cos(2\pi 5t)$. Then we can get similar result: two peaks at 3 and 5 Hz. And now, we can make a bold assumptions that we can use the combination of sine or cosine signal to build another arbitrary signal. And we should proof it.\par
Let us assume $f(t)=\sum\limits^{N}A_n \sin (n\omega t+\phi)$, then we will prove it.
Since we want to using the combination of sine function, we let $f_T(t)$ equals to the sum of n-terms sine signal.
%%Orignal sine formula
$$
f_T(t)=C_0+\sum\limits_{n=1}^N{\sin(n\Omega t+\phi_{n} ) } \quad \Omega=\frac{2\pi}{t}
$$

And we can make some modifications to the formula.
%%Expansion of last formula
$$
f_T(t)=C_0+\sum\limits_{n=1}^N{[
	A_n\sin(n\Omega t)\cos\phi_n+
	A_n\cos(n\Omega t)\sin\phi_n
	]}
$$

Let's denote $a_n=A_n\cos\phi_n$ and $b_n=A_n\sin\phi_n$
%%Denote an and bn
$$
f_T(t)=C_0+\sum\limits_{n=1}^N{[
	a_n\sin(n\Omega t)+
	b_n\cos(n\Omega t)
	]}
$$

Apply Euler's formula
%%Euler's formula
\begin{align}\nonumber
\sin(x)=\frac{e^{jx}-e^{-jx}}{2j}\nonumber \\
\cos(x)=\frac{e^{jx}+e^{-jx}}{2}\nonumber \\
e^{jx}=\cos(x)+j\sin(x)\nonumber
\end{align}
%%Applied euler's formula and merge congeners 
\begin{align}
f_T(t)&=C_0+\sum\limits_{n=1}^{N}{[
	a_n( \frac{ e^{jn\Omega t} + e^{-jn\Omega t} }{2j} )+
	b_n( \frac{ e^{jn\Omega t} + e^{-jn\Omega t} }{2} )
	]}\nonumber \\ 
&=C_0+\sum\limits_{n=1}^{N}{[
	( \frac{b_n-ja_n}{2} )e^{jn\Omega t}+
	( \frac{b_n+ja_n}{2} )e^{-jn\Omega t} 
	]}\nonumber
\end{align}

Then we can denote $C_n=\frac{b_n-ja_n}{2}$ and $C_{-n}=\frac{b_n+ja_n}{2}$
%%Denote Cn
\begin{align}
f_T(t)&=C_0+\sum\limits_{n=1}^{N}{(
	C_n e^{jn\Omega t} + C_{-n} e^{-jn\Omega t}
	)}\nonumber \\
&=\sum\limits_{n=-N}^{N}{C_n e^{jn\Omega t}}\nonumber
\end{align}

Now we have the Fourier series of periodic signal $f_T(t)$, but we don't know how to calculate $C_n$.At this time, we can extract one term from n terms sum.We can denote it as $C_m$.
%%How to calc Cn
\begin{flalign}
&f_T(t)=C_m e^{jm\Omega t} + 
	\sum\limits_{ \substack{n=-N \\ n\neq m} }^{N}{C_n e^{jn\Omega t}} \nonumber \\
&C_m e^{jm\Omega t}=f_T(t)-
	\sum\limits_{ \substack{n=-N \\ n\neq m} }^{N}{C_n e^{jn\Omega t}} \nonumber \\
&C_m=f_T(t) e^{-jm\Omega t}-
	\sum\limits_{ \substack{n=-N \\ n\neq m} }^{N}{C_n e^{j(n-m)\Omega t}} \nonumber
\end{flalign}

We can do integration in both sides
%%Integration of left and right
\begin{align}
&\int_{0}^{T}{C_m\dif t}=\int_{0}^{T}{(f_T(t) e^{-jm\Omega t}-
	\sum\limits_{ \substack{n=-N \\ n\neq m} }^{N}{C_n e^{j(n-m)\Omega t}})
	\dif t} \nonumber \\
&TC_m=\int_{0}^{T}{[f_T(t) e^{-jm\Omega t}-
	\sum\limits_{ \substack{n=-N \\ n\neq m} }^{N}{C_n e^{j(n-m)\Omega t}}]
	\dif t} \nonumber
\end{align}

Since integration and summation is linear operation, we can put integration into summation.
%%Swap integratuion and summation
\begin{align}
	&TC_m=\int_{0}^{T}{f_T(t) e^{-jm\Omega t}} \dif t -
		\sum\limits_{ \substack{n=-N \\ n\neq m} }^{N}{
			\int_{0}^{T}C_n e^{j(n-m)\Omega t} \dif t
		} \nonumber
\end{align}

For the inner item of summation, we have
%%Calculate orthogonal terms 
\begin{align}
&\int_{0}^{T}{C_n e^{j(n-m)\Omega t} \dif t } \nonumber \\
=&\ C_n\int_{0}^{T} e^{j(n-m)\Omega t} \nonumber \\
=&\ \frac{C_n}{j(n-m)\Omega} \Big[ e^{ j(n-m) \frac{2\pi}{T} t} \Big]_{0}^{T} \nonumber \\
=&\ 0 \nonumber
\end{align}

So, we can get
%%Cn result
$$
C_n=\frac{1}{T}\int_0^T{f_T(t) e^{-jn\Omega t} \dif t}
$$

Now, we can get a series which is contain the combination of sine and cosine function. We can simply use it to verify our suggestion. If we have a square wave with frequency of 1Hz and 50\% duty cycle. Let's calculate its Fourier coefficient. And we should always start with $C_0$.
%%C0 of square wave
\begin{align}
C_0 &= \ \int_{0}^{1}{f_T(t) e^{-j\Omega t \cdot 0} \dif t} \quad \Omega=\frac{2\pi}{T} \nonumber \\
	&=\ \int_{0}^{1}{f_T(t) \dif t} \nonumber \\
	&=\ \int_{0}^{0.5}{\dif t} + \int_{0.5}^{1}{- \dif t} \nonumber \\
	&=\ 0 \nonumber \\
C_n &= \ \int_{0}^{1}{ f_T(t) e^{-jn\Omega t} \dif t} \nonumber \\
	&= \ \int_{0}^{0.5}{e^{-jn\Omega t} \dif t} +
		 \int_{0.5}^{1}{-e^{-jn\Omega t} \dif t} \nonumber \\
	&= \ \Big[-\frac{1}{jn\Omega} e^{-jn\Omega t}\Big|_{0}^{0.5} \Big] -
	   	 \Big[\frac{1}{jn\Omega} e^{-jn\Omega t}\Big|_{0.5}^{1} \Big] \nonumber
\end{align}
Apply $\Omega \ = \ \frac{2\pi}{T} $
\begin{align}
C_n	&= \ \Big[-\frac{1}{jn2\pi} e^{-jn2\pi t}\Big|_{0}^{0.5} \Big] -
		 \Big[\frac{1}{jn2\pi} e^{-jn2\pi t}\Big|_{0.5}^{1} \Big] \nonumber \\
	&= \ \cfrac{j}{2\pi n}\Big[ e^{-j\pi n} -1 -e^{-j2\pi n} +e^{-j\pi n} \Big] \nonumber \\
	&= \ \cfrac{j}{2\pi n}\Big[ 2e^{-j\pi n -1 -e^{-j2\pi n}} \Big] \nonumber
\end{align}
We can simply calculate that $e^{-j2\pi n}=1$ and $e^{-j\pi n}=(-1)^{n}$. So we can get
\begin{align}
C_n &= \ \cfrac{j}{\pi n}\Big[ (-1)^n-1 \Big] \nonumber \\
	&=\begin{cases}
		\cfrac{-2j}{\pi n}&  {odd \ n} \\
		0&  {even \ n}
 	  \end{cases}	\nonumber
\end{align}
And the Fourier series of square wave will be
$$
\mathscr{F}_f(t)= \ \cfrac{4}{\pi} \sum\limits_{k=1}^K{\cfrac{1}{2k-1} \sin [2\pi (2k+1)t]}
$$
\begin{figure}[H]
	\centerline{\animategraphics[autoplay,loop,scale=0.5]{1}{figure/fig_5/fig-}{0}{5}}
\end{figure}
\clearpage

\section{\sc II. When the infinite sum has to be applied}
Now, we have an almost-Fourier series. But we can find ,when we use sine wave to build square wave, this sequence converges to the square wave except at the point $x_0 = 0$, which a point of discontinuity of it. In 1898, American Albert Michelson made a harmonic analyzer. When he tested the square wave, he was surprised to find that the $X_n(T)$ of the square wave jumped near the discontinuous points, and the peak size of the fluctuation did not seem to decrease with the increase of n! So he wrote to Gibbs, the famous mathematical physicist at that time. Gibbs got interested to the behavior of the sequence of Fourier partial sums around these points. \par
Indeed, when x is close to the point 0, the graphs present a bump. Let us do some calculations to justify this phenomenon. Consider the derivative of $\mathscr{F}_f(t)$.
\begin{align}
\mathscr{F}^{'}_f(t) 
	&= \ \cfrac{4}{\pi} \sum\limits_{n=0}^N{\cos [2\pi (2k+1)t]} \nonumber \\
	&= \ \cfrac{4}{\pi} \left[ \cos(t) + \cos(3t) + cos(5t) + \cdots + cos((2N-1)t) \right] \nonumber 
\end{align}
Using trigonometric identities, we get 
\begin{align}
\pi\sin(t) \mathscr{F}^{'}_f(t) 
	&= \ 2\left[ \sin(2t) + (\sin(4t)-\sin(2t)) + (\sin(6t)-\sin(4t)) + \cdots + (\sin(2nt)-\sin((2n-2)t))\right] \nonumber \\
	&= \ 2\sin(2nt) \nonumber
\end{align}
So the critical points of $\mathscr{F}_f(t)$ are 
$$
2nt = \pm\pi , \pm 2\pi,\cdots,\pm (2n-1)\pi
$$
Since the functions are odd, we will only focus on the behavior to the right of 0. The closest critical point to the right of 0 is $\frac{\pi}{2n}$. Hence 
$$
\mathscr{F}_f(\cfrac{\pi}{2n}) = \cfrac{4}{\pi} {
	\sin(\frac{\pi}{2n}) + \cfrac{\sin(\frac{3\pi}{2n})}{3} + \cdots +
	\cfrac{\sin(\frac{(2n-1)\pi}{2n})}{2n-1}
}
$$
In order to find the asymptotic behavior of the this sequence, when n is large, we will use the Riemann sums. Indeed, consider the function $Sa(t) = \frac{\sin(t)}{t}$ on the interval $[0,\pi]$, and the partition $\left\{{\frac{k\pi}{n}}, \;\;k \in [1,n]\right\}$ of $[0,\pi]$. So the Riemann sums
$$
\frac{\pi}{n}
\left(
	\cfrac{\sin(\cfrac{\pi}{2n})}{\cfrac{\pi}{2n}} + \cdots +
	\cfrac{\sin(\cfrac{(2n-1)\pi}{2n})}{\cfrac{(2n-1)\pi}{2n}}
\right)
$$
converges to $\int_{0}^{\pi} Sa(t) \dif t$. Easy calculations show that these sums are equal to 
$$
\cfrac{\pi}{2}{\mathscr{F}_f \left( \cfrac{\pi}{2n} \right) }
$$

Hence

$$
\lim_{n \rightarrow \infty} \mathscr{F}_f \left(\frac{\pi}{2n}\right) = \frac{2}{\pi}\int_{0}^{\pi}Sa(t) \dif t.
$$

Using Taylor polynomials of $\sin(t)$ at 0, we get

$$
\frac{2}{\pi}\int_{0}^{\pi}\frac{\sin(t)}{t} \dif t = 2 - \frac{\pi^2}{9} + \frac{\pi^4}{300} - \frac{\pi^6}{17640}+\ldots
$$

i.e. up to two decimals, we have

$$
\frac{2}{\pi}\int_{0}^{\pi}\frac{\sin(t)}{t} \dif t = 1.18
$$
These bumps seen around 0 are behaving like a wave with a height equal to 0.9045. This is not the case only for this function. Indeed, Gibbs showed that if f(t) is piecewise smooth on $[-\pi,\pi]$, and $t_0$ is a point of discontinuity, then the Fourier partial sums will exhibit the same behavior, with the bump's height almost equal to 
$$
0.09\left[ f(t_0+) + f(t_0-)\right]
$$
\clearpage

\section{\sc III. Open the gate to frequency domain}
\clearpage

\section{\sc IV. From periodicity to aperiodicity}
\clearpage

\end{resume}
\end{document}




